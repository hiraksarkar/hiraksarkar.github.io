% LaTeX file for resume 
% This file uses the resume document class (res.cls)

\documentclass{res}
\RequirePackage{marvosym}
\usepackage{color} 
\usepackage{xspace}
\usepackage{hyperref}
\usepackage{comment}
\usepackage{academicons}


\setlength{\textheight}{9in} % increase text height to fit on 1-page 
\newcommand{\latex}{\LaTeX\xspace}

\begin{document} 

 %\moveleft.5\hoffset\centerline{\LARGE\bf Hirak Sarkar}
 \moveleft\hoffset\leftline{\LARGE\bf Hirak Sarkar}
% \name{HIRAK SARKAR\\[12pt]}     % the \\[12pt] adds a blank
				        % line after name  
%\moveleft\hoffset\vbox{\hrule width\resumewidth height 1pt}

\moveleft\hoffset\vbox{\hrule width\resumewidth height 0.7pt}    
\vspace{-0.5cm}
\address{5812 Quebec St. \\Berwyn Heights, MD-20740\\Mob No.+1 6315208131}

\address{ \Letter \hspace{1.0mm}\href{mailto:hsarkar@cs.stonybrook.edu}{hsarkar@cs.umd.edu}  \\ 
           {\Large\ComputerMouse} \hspace{0.4mm}  \href{www.hiraksarkar.com}{www.hiraksarkar.com}\\
           {\aiGoogleScholar} \hspace{0.6mm}  \href{https://tinyurl.com/yd3bltxg}{Google Scholar}\\
        }  
                                  
\begin{resume}
%\section{Research Interest}
% \vspace{0.1in}
%I am interested in applying machine learning techniques such as statistical inference and deep learning to analyse and extract information from  big data in the field of genomics, social and computer networks.

%\vspace{-0.2in}
\section{Education}          
  \vspace{-0.1in}	
   \begin{tabbing}
   \hspace{2.3in}\= \hspace{2.6in}\= \kill % set up two tab positions
    {\bf Ph.D in Computer Science} \>   \>  2020 (expected)
   \end{tabbing}  \vspace{-20pt}      % suppress blank line after tabbing
  % {\bf Bachelor of Technology(Computer Science and Engineering)}  \\
      Efficient Processing and Statistical Inference in RNA-seq data, {\it advisor: Prof. Rob Patro}  \\        
       University of Maryland, MD     
       
\vspace{-0.1in}	
   \begin{tabbing}
   \hspace{2.3in}\= \hspace{2.6in}\= \kill % set up two tab positions
     {\bf MS in Computer Science} \>    \>  2014 - transferred
   \end{tabbing}  \vspace{-20pt}      % suppress blank line after tabbing
  % {\bf Bachelor of Technology(Computer Science and Engineering)}  \\
        Improving storage and alignment methodologies for RNA-seq data, {\it advisor: Prof. Rob Patro}  \\        
       Stony Brook University, NY     \\
       GPA: 3.9/4.00
 
 \vspace{-0.1in}	
   \begin{tabbing}
   \hspace{2.3in}\= \hspace{2.6in}\= \kill % set up two tab positions
     {\bf Masters of Technology (Computer Science)}  \>     \>2011-2013 
   \end{tabbing}  \vspace{-20pt}      % suppress blank line after tabbing
  % {\bf Bachelor of Technology(Computer Science and Engineering)}  \\        
       Indian Statistical Institute, Calcutta   \\
       $1^{st}$ Class (Hons.) 


\vspace{-0.1in}	
\begin{tabbing}
\hspace{2.3in}\= \hspace{2.6in}\= \kill % set up two tab positions
{\bf Bachelor of Technology (Computer Science and Engineering)}  \>     \>2007-2011
\end{tabbing}  \vspace{-20pt}      % suppress blank line after tabbing
  % {\bf Bachelor of Technology(Computer Science and Engineering)}  \\        
West Bengal University of Technology     \\       
GPA: 8.88/10      \\   
%B.Tech Thesis Topic: {\bf \color{blue} \underline {GameSAT: A Structured Approach to Combine SLS SAT Solvers}}  \\

\vspace{-0.2cm}
\section{Publications (Bioinformatics)}
%\subsection{Bioinformatics}
\begin{enumerate}

\item {\it Terminus enables the discovery of data-driven, robust transcript groups from RNA-seq data.}, by \underline{Hirak Sarkar}, Avi Srivastava, Hector Corrada Bravo, Michael I. Love and Rob Patro. [\textit{\textbf{ISMB' 20}}]

\item {\it A Bayesian framework for inter-cellular information sharing improves dscRNA-seq quantification}, by Avi Srivastava, Laraib Malik, \underline{Hirak Sarkar}, Rob Patro. [\textit{\textbf{ISMB' 20}}]

\item {\it Alignment and mapping methodology influence transcript abundance estimation}, by Avi Srivastava, Laraib Malik, \underline{Hirak Sarkar}, Mohsen Zakeri, Charlotte Soneson, Michael I. Love, Carl Kingsford, Rob Patro. [Submitted  \textit{\textbf{Genome Bio.}}]

\item {\it Minnow: A principled framework for rapid simulation of dscRNA-seq data at the read level}, by \underline{Hirak Sarkar}, Avi Srivastava and Rob Patro [\textit{\textbf{ISMB'19}}]. 

\item {\it Towards Selective-Alignment: Producing Accurate And Sensitive Alignments Using Quasi-Mapping}, by \underline{Hirak Sarkar*}, Mohsen Zakeri*, Laraib Malik and Rob Patro [\textit{\textbf{ACM BCB'18}}]. 

\item {\it An Efficient, Scalable and Exact Representation of High-Dimensional Color Information Enabled via de Bruijn Graph Search}, by Fatemeh Almodaresi*, \underline{Hirak Sarkar*}, Avi Srivastava and Rob Patro [\textit{\textbf{ISMB'18}}]. 

\item   {\it Quark enables semi-reference-based compression of RNA-seq data} by  \underline{Hirak Sarkar} and Rob Patro [\textit{accepted \textbf {Bioinformatics'17}}].


\item   {\it Fast, Lightweight Clustering of de novo Transcriptomes using Fragment Equivalence Classes} by A Srivastava*, \underline{Hirak Sarkar*}, Laraib Malik and Rob Patro (* \textit{Joint first authors}) [\textit{\textbf{RECOMB-seq'16}}]. 


\item {\it RapMap: A Rapid, Sensitive and Accurate Tool for Mapping RNA-seq Reads to Transcriptomes} by A Srivastava, \underline{Hirak Sarkar}, Nitish Gupta and Rob Patro  [\textit{\textbf{ISMB'16}}].

%\item {\it Voronoi Game on Graphs} by  S. Bandyapadhyay, A. Banik, S. Das and \underline{H. Sarkar} (in alphabetical order of surnames) Seventh International Workshop on Algorithms and Computation. \textit{\textbf{WALCOM'13}}.
%\vspace{-0.5cm}
\end{enumerate}
\section{Non-Bioinformatics}
\begin{enumerate}
  \item {\it Social Media Attributions in the Context of Water Crisis} by Rupak Sarkar, \underline{Hirak Sarkar}, S Mahinder and AR KhudaBukhsh.  [\textit{\textbf{ArXiv'20}}] 
  \item {\it Voronoi Game on Graphs} (Extended version) by S. Bandyapadhyay, A. Banik, S. Das and \underline{H. Sarkar} (in alphabetical order of surnames) {\it Journal of Theoretical Computer Science}  [\textit{\textbf{Journal of Theoretical Computer Science'15}}].
\end{enumerate}
\vspace{-0.5cm}
\section{Posters}
\begin{enumerate}
\item {\it Pufferfish: A fast graph-based indexing and query strategy for large genomic sequences} by Fatemeh Almodaresi*, \underline{Hirak Sarkar*}, and Rob Patro, Poster presented in [\textit{\textbf{WABI'17}}].

\item {\it Joint probabilistic model for multiple steps of gene regulation} by \underline{Hirak Sarkar}, Yi-Fei Huang and Adam Siepel, Poster presented in  \textit{\textbf{BioData'16}}
\vspace{-0.5cm}
\end{enumerate}


\section{Professional Experience}
\begin{itemize}
\item {\textbf {Facebook Inc.}} Worked as Ph.D data scientist intern. I worked on designing scalable pipelines to analyze the data from millions of users. Applied transfer learning based methods in order to measure the effect of natural calamity such as cyclone or wild fire from open source satellite images.   

\item {\textbf {Simons Center for Quantitative Biology, Cold Spring Harbor Lab:}} Worked under the supervision of Prof. Adam Seipel from May, 2016 to July, 2016. We designed probabilistic graphical model to infer transcription and degradation rates from different assays such as GRO-seq and RNA-seq.  

\item {\it Summer Assistantship '15,'17} with Prof. Rob Patro. We worked on various problems ranging from 
\item  Teaching Assistant for CSE549 (Computational Biology), CSE219 (Game Programming)
\item {\it Visiting Researcher} at Advanced Computing \& Microelectronics Unit, Indian Statistical Institute from October, 2013 to December, 
2013. I worked on Computational Geometry and Graph Theory
\item {\it Junior Research Fellow} in Department of Computer Science \& Engineering at Indian Institute of Technology, Kharagpur (IIT) 
from July, 2013 to Sept, 2013. I was a member of Complex Network Engineering Group. I did TA-ship for Introductory Programming 
Course in that brief stint. 
\end{itemize}

% \section{Relevant Course Projects}
% \begin{itemize}

%  \item{ \it{IPID Header Survey:}} We used IPID headers to estimate the load over different servers, sampled from alexa
%  list. The main challenge of the project is to detect the wrapping pattern and navigate through the global vs local IPID counter. 
%  We also looked at the temporal pattern of network traffic for the different regional websites which shows interesting correlation with possible 
%  working load at the server end. \\
%  {\it Instructor: Prof. Phillipa Gill}
 
%  \item {{\it Some Geometric and Combinatorial Properties of Binary Matrices Related to
% Discrete Tomography:}} Here we are trying to decompose an image matrix into matrices
% each having orthogonal convex polygon also known as Ferrer?s digraph. An operation could
% regenerate the original image from these matrices. The methods can be applied to image and
% data compression. (\href{https://drive.google.com/open?id=0B3ErYrn4jOcxSHhReVF6cmROelE}{\it  \underline{Masters dissertation}}) 
% {\it Advisor: Prof. Bhargab B Bhattacharya \& Prof. Sandip Das}

%  \item 
%  {{\it GameSAT- A Structured Approach to Combine SLS SAT Solvers:}} Here we used several existing heuristic algorithms to mix up with each other in a customized probabilistic fashion in order to solve combinatorial hard problems encoded as SAT instances. \\
% ({\it B.Tech dissertation}) {\it Advisor: Dr. Ashiqur KhudaBukhsh, CMU}
 
%  \end{itemize}
 
 
\section{Awards and Honors}
   \begin{itemize}
   \item Awarded {\it Research Assistantship}, SBU ({\it 2016-present})
   \item Awarded {\it Special CS Chair Fellowship} ({\it of \$10000} ), SBU ({\it 2014-2015})
   \item Awarded {\it NUS Research Scholarship}, NUS ({\it Jan'14-June'14})
   \item Awarded {\it Post-graduate Scholarship} by, Govt. of India. ({\it 2011-2013}) 
   \item Received {\color{blue} First Prize} for Software Competition (IEM), Calcutta.
   \end{itemize}

\section{Programming Skills}
Python, C++, Rust

% \section{\bf Open Source Tools Used}
% Dendropy, BioNet  (Comp Bio) \\
% NLTK, Scrapy, Scikit-learn, Stanford Parser, Pandas (Data Science) \\



% \section{Relevant Coursework} 
% \begin{itemize}
% \item Artificial Intelligence,  Computational Biology, Analysis of Algorithms, Fundamental of Networks. (at {\it SBU})
% \item Machine Learning \& Pattern Recognition, Image Processing, Stochastic Process, Optimization Algorithms, Computer Graphics. (at {\it
% Indian Statistical Institute})
% \end{itemize}

\section{References}
\begin{itemize}
 \item Prof. Robert Patro 
 \item Prof. Michael I. Love
 \item Prof. He\'ctor Corrada Bravo
\end{itemize}

\end{resume}
\end{document}
